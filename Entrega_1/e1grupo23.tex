\documentclass{article}

\usepackage[margin=1in]{geometry}
\usepackage{amsmath,amssymb}
\usepackage{multicol}
\usepackage{float}
\usepackage{adjustbox}
\usepackage[spanish]{babel}
\usepackage[utf8]{inputenc}

\usepackage{graphicx}
\graphicspath{ {/Users/imagesLatex/} }

\begin{document}

\noindent
\begin{tabular*}{\textwidth}{l @{\extracolsep{\fill}} r @{\extracolsep{4pt}} l}
\text{} & \text{DEPARTAMENTO DE CIENCIA DE LA COMPUTACI\'ON}\\
\end{tabular*}\\
\rule[2ex]{\textwidth}{0.5pt}
\text{Base de Datos - IIC2413}\\
\text{Nombre Alumnos: Tanya Garrido - Rodrigo Zapata S.}\\
\text{Profesor: J. Reutter}\\
\text{Fecha: } \today\\
\rule[2ex]{\textwidth}{0.5pt}
\begin{center}
\Large\textbf{Tarea 1}\\
\end{center}

\noindent
\textbf{Parte 1: Descripci\'on del esquema}

Acontinuaci\'on se detallan las tablas creadas y sus respectivos atributos. El nombre del atributo se resaltar\'a con negrita y si el atributo es una llave, se resaltar\'a con un subrayado.

$ \\ $
\textbf{Usuarios}
\begin{itemize}
	\setlength{\itemindent}{.5in}
	\item{\underline{\textbf{id}} integer}
	\item{\textbf{nombre} varchar(30)}
	\item{\textbf{telefono} integer}
	\item{\textbf{mail} varchar(30)}
	\item{\textbf{pais} varchar(20)}
	\item{\textbf{saldo} float: cuantos zorzales posee la persona}
\end{itemize}

$ \\ $
\textbf{Transacciones}
\begin{itemize}
	\setlength{\itemindent}{.5in}
	\item{\textbf{rec\_id} integer: id de persona quien recibe el dinero}
	\item{\textbf{manda\_id} integer: id de quien transfiere dinero}
	\item{\underline{\textbf{id\_trans}} integer: id de la transacci\'on}
	\item{\textbf{fecha} varchar(20)}
	\item{\textbf{cantidad} float: cantidad de zorzales transferidos}
\end{itemize}

$ \\ $
\textbf{PreciosHistoricos}
\begin{itemize}
	\setlength{\itemindent}{.5in}
	\item{\textbf{\underline{fecha}} varchar(20)}
	\item{\textbf{z\_en\_usd} float: a cuantos dolares corresponde un zorzal}
	\item{\textbf{z\_en\_clp} float: a cuantos pesos chilenos corresponde un zorzal}
\end{itemize}
	

\newpage
\textbf{Exchange}
\begin{itemize}
	\setlength{\itemindent}{.5in}
	\item{\textbf{\underline{id}} integer}
	\item{\textbf{rec\_id} integer}
	\item{\textbf{manda\_id} integer}
	\item{\textbf{fecha} varchar(20)}
	\item{\textbf{cant\_transf} float}
	\item{\textbf{divisa} varchar(3): con que moneda paga quien transfiere, USD o CLP}
\end{itemize}

$ \\ $
$ \\ $
\noindent
\textbf{Parte 2} \\
\textbf{Consultas}

$ \\ $
	Dado un usuario (X) y un d\'ia (AAAA-MM-DD), liste todas las transacciones de ese usuario
\begin{center}
	$ \pi_{cantidad}[ \sigma_{nombre = X \wedge fecha = AAAA-MM-DD}( Usuarios \bowtie_{id = manda\_id} Transacciones) ] $
\end{center}

$ \\ $
	Dado un usuario (X), muestre su \'ultima transacci\'on
\begin{center}
	$ \rho (R1, \pi_{fecha}(\sigma_{nombre = X}(Usuarios \bowtie_{id = manda\_id Transacciones}) )) $ \\
	
	$ \rho (R2, \pi_{fecha}(\sigma_{nombre = X}(Usuarios \bowtie_{id = manda\_id Transacciones}) )) $ \\
	
	$ \pi_{fecha}[ R1 - \pi_{R1.fecha}( \sigma_{R1.fecha < R2.fecha}( R1 \times R2 ) ) ] $
\end{center}
\end{document}